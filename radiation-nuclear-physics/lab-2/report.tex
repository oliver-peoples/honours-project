\documentclass[a4paper,twocolumn]{IEEEtran}

\usepackage{amsmath,amsfonts,amssymb}
\usepackage{graphicx,float,subfig}
\usepackage{booktabs}
\usepackage[colorlinks=true]{hyperref}

\author{Oliver Kirkpatrick\footnote{oliver.kirkpatrick@rmit.edu.au}}

\begin{document}
    \title{Analysis of Gamma Ray Spectra from Reference Isotopes with Multi-Channel Analyzers}
    \author{\IEEEauthorblockA{Oliver Kirkpatrick}
    \\
    \IEEEauthorblockA{School of Engineering, RMIT University\\
    Email: s3725341@student.rmit.edu.au}}

    \maketitle
    \begin{abstract}
        stuff
    \end{abstract}
    \section{Results and Discussion}
    \subsection{Calibration Process}
    \begin{figure}[H]
        \centering
<<<<<<< HEAD
        \subfloat[Recording 1, hh:mm:ss, $T+$ minutes]{\includegraphics[width=0.95\linewidth]{figures/calibration_spectrum_cs_counts_overlay_both_mca.png}}\\
        \subfloat[Recording 2, hh:mm:ss, $T+$ minutes]{\includegraphics[width=0.95\linewidth]{figures/calibration_spectrum_na_counts_overlay_both_mca.png}}\\
        \subfloat[Recording 3, hh:mm:ss, $T+$ minutes]{\includegraphics[width=0.95\linewidth]{figures/calibration_spectrum_co_counts_overlay_final_mca.png}}
=======
        \subfloat[$^{137}$Cs Gamma Ray Spectra\label{subfig:cs-137-grs}]{\includegraphics[width=0.95\linewidth]{figures/calibration_spectrum_cs_counts_overlay_both_mca.png}}\\
        \subfloat[$^{22}$Na Gamma Ray Spectra\label{subfig:na-22-grs}]{\includegraphics[width=0.95\linewidth]{figures/calibration_spectrum_na_counts_overlay_both_mca.png}}\\
        \subfloat[$^{60}$Co Gamma Ray Spectra\label{subfig:co-60-grs}]{\includegraphics[width=0.95\linewidth]{figures/calibration_spectrum_co_counts_overlay_final_mca.png}}
>>>>>>> 47bc9b3a34dcb9636449b8798d905a01d4e4a48e
        \caption{Gamma ray spectra for known isotopes of Cesium ($^{137}$Cs), Sodium ($^{22}$Na), and Cobalt ($^{60}$Co). Cobalt was the final isotope tested, and the calibration metrics calculated for it were used for the rest of the experiment. For the Cesium and Sodium isotopes, the spectra with the calibration at recording (\textbf{C.A.R.}), is shown in solid black, and the spectra adjusted with the final calibration (\textbf{F.C.}) is shown in dashed gray. Raw counts for all isotopes are shown in blue, though these are often obscurred by the $\sigma=3$ gaussian smoothed curves.}
    \end{figure}
    \subsection{Background Radiation}
    \begin{figure}[H]
        \centering
        \includegraphics[width=0.95\linewidth]{figures/background_counts_overlay.png}
        \caption{Background radiation profile.}
    \end{figure}
    \subsection{Analysis of Unknown Sample}
    \begin{figure}[H]
        \centering
        \includegraphics[width=0.95\linewidth]{figures/difference_counts_smooth.png}
    \end{figure}

    \begin{table*}[t]
        \centering
        \caption{Total counts and peak energies with with increasing time since first measurement.}
        \begin{tabular}{c c c c c c c}
            \toprule
            \textbf{Sample Number} & $T+$ \textbf{(s)} & \textbf{Total Counts} & \textbf{Peak 1 Energy} & \textbf{Peak 2 Energy} & \textbf{Peak 3 Energy} & \textbf{Peak 4 Energy} \\
            \midrule
            1 & 0s & 68493 & 401.24 keV &  815.92 keV & 1096.59 keV & 1296.46 keV \\
            2 & 420s & 63755 & 400.52 keV & 799.32 keV & 1106.59 keV & 1295.25 keV \\
            3 & 819s & 59850 & 401.25 keV & 796.80 keV & 1099.32 keV  & 1293.16 keV \\
            4 & 1199s & 56863 & 400.35 keV & 759.04 keV & 1097.07 keV & 1295.73 keV \\
            5 & 1559s & 53111 & 401.94 keV & 805.54 keV & 1104.55 keV & 1296.78 keV \\
            6 & 1936s & 50483 & 400.41 keV & 812.17 keV & 1085.92 keV & 1292.84 keV \\
            \bottomrule
        \end{tabular}
        \label{tab:counts_v_time}
    \end{table*}
    \onecolumn
    \appendix
    \begin{figure}[H]
        \centering
        \subfloat[Recording 1, hh:mm:ss, $T+$ minutes\label{subfig:mystery-recording-1}]{\includegraphics[width=0.49\textwidth]{figures/mystery_1_background_counts_overlay.png}}
        \hspace{\fill}
        \subfloat[Recording 2, hh:mm:ss, $T+$ minutes\label{subfig:mystery-recording-2}]{\includegraphics[width=0.49\textwidth]{figures/mystery_2_background_counts_overlay.png}}\\
        \subfloat[Recording 3, hh:mm:ss, $T+$ minutes\label{subfig:mystery-recording-3}]{\includegraphics[width=0.49\textwidth]{figures/mystery_3_background_counts_overlay.png}}
        \hspace{\fill}
        \subfloat[Recording 4, hh:mm:ss, $T+$ minutes\label{subfig:mystery-recording-4}]{\includegraphics[width=0.49\textwidth]{figures/mystery_4_background_counts_overlay.png}}\\
        \subfloat[Recording 5, hh:mm:ss, $T+$ minutes\label{subfig:mystery-recording-5}]{\includegraphics[width=0.49\textwidth]{figures/mystery_5_background_counts_overlay.png}}
        \hspace{\fill}
        \subfloat[Recording 6, hh:mm:ss, $T+$ minutes\label{subfig:mystery-recording-6}]{\includegraphics[width=0.49\textwidth]{figures/mystery_6_background_counts_overlay.png}}
        \caption{Raw count data for each recording of gamma rays from the unknown isotope.}
    \end{figure}
    ss
\end{document}