\documentclass{article}

\title{Gamma-ray spectra: Pulse Height Analysis using a
Single Channel Analyser (SCA)}

\author{
    Oliver Kirkpatrick\footnote{s3725341@student.rmit.edu.au}\\
    Pramod Joshi\footnote{s3989210@student.rmit.edu.au}
}

\begin{document}

\maketitle
\begin{abstract}
    The aim of this experiment was to utilize a Single Channel Analyser (SCA) to perform pulse height analysis and obtain a gamma ray energy spectrum of a radioisotope. The NIM system was used to produce signal pulses in response to radiation detection events, where the voltage amplitude of a pulse was proportional to the energy of the gamma ray that generated it. %By measuring the amplitudes of a series of pulses, an energy spectrum of gamma emissions from a radioisotope was produced. The experiment demonstrated that the SCA and NIM system can be effectively used to obtain a gamma ray energy spectrum of a radioisotope by utilizing pulse height analysis.
\end{abstract}
\section{Pre-lab notes}
\begin{itemize}
    \item The SCA allows the user to determine the amplitude of each pulse within a user-set range
    \item The number of counted pulses within adjacent intervals provides a pulse height spectrum
    \item The decay scheme for $^{137}Cs$ involves a gamma ray emitted with an energy of 662 keV, with beta emissions typically ignored
    \item The two unknowns of interest for gamma sources are the energies of the emitted gammas and the rate of gamma emission, which can identify the isotope and relate to the amount present, respectively
    \item The experiment aims to introduce basic measurements associated with gamma emitting sources, with Experiment 2 dealing with interactions of gammas in the detector medium and their impact on the spectrum obtained.
\end{itemize}
\section{During Lab Notes}
\begin{itemize}
    \item 
\end{itemize}

\end{document}